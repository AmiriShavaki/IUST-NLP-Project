\Section
{آموزش و تحلیل بردار کلمات (\lr{word2vec})}
{
در این‌جا معیار بکارگرفته‌شده برای شباهت‌سنجی بین بردارها، شباهت کسینوسی بوده است که از رابطه
$\frac{A.B}{|A||B|}$
بدست می‌آید.

\subsection{شبیه‌ترین کلمات مشترک بین جفت لیبل‌ها}
{

\begin{adjustbox}{width=\textwidth}
  \csvautotabular{Tables/word2vec_max_com_sim.csv}  
\end{adjustbox}

}
\subsection{متفاوت‌ترین کلمات مشترک بین جفت لیبل‌ها}
{

\begin{adjustbox}{width=\textwidth}
  \csvautotabular{Tables/word2vec_min_com_sim.csv}  
\end{adjustbox}
$\quad$

نکته جالب تفاوت کمتر بردارها در جفت (۱ستاره، ۵ستاره) نسبت به بقیه جفت‌ها است. در این جفت حداکثر تفاوت برداری $0.17$ بوده در حالی که همانطور که در جدول می‌بینید در بقیه جفت‌ها میزان تفاوت متفاوت‌ترین بردارها بیشتر است.

کلمه‌ 
\lr{"tnx"}
در جفت (۵ستاره و ۲ستاره) بیشترین تفاوت برداری را داشته است. احتمالا بخاطر اینکه در داده‌های ۵ستاره این کلمه همراه با رضایت‌مندی و خوشحالی کاربر بیان شده در حالی که در کلاس ۲ستاره این کلمه توأم با نارضایتی و شاید با لحن کنایه‌آمیز بیان شده است و تفاوت برداری احتمالا ناشی از همین تفاوت معنی در بکارگیری کلمه بوده است.

در جفت 
(۳ستاره و ۴ستاره)
کلمه ی \lr{"reccomended"}
که دارای غلط املایی است دارای بیشترین تفاوت برداری است. احتمالا تعداد رخ دادن این کلمه با این غلط املایی خاص در بین داده‌ها کم بوده و وقتی بطور تصادفی در هر دو کلاس ۳ستاره و ۴ستاره از این غلط املایی یافت شده، مدل تعداد زیادی از این کلمه با این شکل نگارشی را در این دو کلاس ندیده که بتواند تشابهات معنایی آن‌ها را درک کند و احتمالا در همین تعداد محدودی که رخ داده هم \lr{context} معنایی متن در دو کلاس متفاوت بوده.

}
}